% \iffalse meta-comment
%<*internal>
\iffalse
%</internal>
%<*readme>
----------------------------------------------------------------
catcode --- Macros for dealing with TeX catcodes
Author:  Mike Conley
E-mail:  conasdf@gmail.com
License: Released under the LaTeX Project Public License v1.3c or later
See:     http://www.latex-project.org/lppl.txt
----------------------------------------------------------------

Some text about the package: probably the same as the abstract.
%</readme>
%<*internal>
\fi
\def\nameofplainTeX{plain}
\ifx\fmtname\nameofplainTeX\else
  \expandafter\begingroup
\fi
%</internal>
%<*install>
\input docstrip.tex
\keepsilent
\askforoverwritefalse
\preamble
----------------------------------------------------------------
catcode --- Macros for dealing with TeX catcodes
Author:  Mike Conley
E-mail:  conasdf@gmail.com
License: Released under the LaTeX Project Public License v1.3c or later
See:     http://www.latex-project.org/lppl.txt
----------------------------------------------------------------

\endpreamble
\postamble

Copyright (C) 2015 by Mike Conley <conasdf@gmail.com>

This work may be distributed and/or modified under the
conditions of the LaTeX Project Public License (LPPL), either
version 1.3c of this license or (at your option) any later
version.  The latest version of this license is in the file:

http://www.latex-project.org/lppl.txt

This work is "maintained" (as per LPPL maintenance status) by
Mike Conley.

This work consists of the file catcode.dtx and a Makefile.
Running "make" generates the derived files README, catcode.pdf and catcode.sty.
Running "make inst" installs the files in the user's TeX tree.
Running "make install" installs the files in the local TeX tree.

\endpostamble

\usedir{tex/latex/catcode}
\generate{
  \file{\jobname.sty}{\from{\jobname.dtx}{package}}
}
%</install>
%<install>\endbatchfile
%<*internal>
\usedir{source/latex/catcode}
\generate{
  \file{\jobname.ins}{\from{\jobname.dtx}{install}}
}
\nopreamble\nopostamble
\usedir{doc/latex/catcode}
\generate{
  \file{README-\jobname.txt}{\from{\jobname.dtx}{readme}}
}
\ifx\fmtname\nameofplainTeX
  \expandafter\endbatchfile
\else
  \expandafter\endgroup
\fi
%</internal>
% \fi
%
% \iffalse
%<*driver>
\ProvidesFile{catcode.dtx}
%</driver>
%<package>\NeedsTeXFormat{LaTeX2e}[1999/12/01]
%<package>\ProvidesPackage{catcode}
%<*package>
    [2015/07/16 v1.00 Macros for dealing with TeX catcodes]
%</package>
%<*driver>
\documentclass{ltxdoc}
\usepackage[a4paper,margin=25mm,left=50mm,nohead]{geometry}
\usepackage[numbered]{hypdoc}
\usepackage{\jobname}
\EnableCrossrefs
\CodelineIndex
\RecordChanges
\begin{document}
  \DocInput{\jobname.dtx}
\end{document}
%</driver>
% \fi
%
% \GetFileInfo{\jobname.dtx}
% \DoNotIndex{\catcode, \chardef, \endinput, \newcommand}
%
%\title{\textsf{catcode} --- Macros for dealing with \TeX\ catcodes\thanks{This file
%   describes version \fileversion, last revised \filedate.}
%}
%\author{Mike Conley\thanks{E-mail: conasdf@gmail.com}}
%\date{Released \filedate}
%
%\maketitle
%
%\changes{v1.00}{2015/07/16}{First public release}
%
% \begin{abstract}
% ==== Put abstract text here. ====
% \end{abstract}
%
% \section{Usage}
%
% ==== Put descriptive text here. ====
%
% \subsection{Symbolic Names for \TeX's Category Codes}
%
% These can help make the intent of code that deals with catcodes clearer.
%
%
% \newcommand{\catcodeof}{Category code \cc, the catcode of}
% \newcounter{currentcharcode}
% \newcommand{\cc}{\thecurrentcharcode\stepcounter{currentcharcode}}
%
% \DescribeMacro{\catescape}
% \catcodeof\ escape characters.
%
% \DescribeMacro{\catbgroup}
% \catcodeof\ characters that begin a group.
%
% \DescribeMacro{\categroup}
% \catcodeof\ characters that end a group.
%
% \DescribeMacro{\catmathshift}
% \catcodeof\ characters that toggle math mode.
%
% \DescribeMacro{\catalign}
% \catcodeof\ characters used to align text.
%
% \DescribeMacro{\catendline}
% \catcodeof\ end of line characters.
%
% \DescribeMacro{\catparam}
% \catcodeof\ characters used as sigils to identify macro parameters.
%
% \DescribeMacro{\catsuper}
% \catcodeof\ characters used to start superscripting.
%
% \DescribeMacro{\catsub}
% \catcodeof\ characters used to start subscripting.
%
% \DescribeMacro{\catignore}
% \catcodeof\ characters \TeX\ ignores.
%
% \DescribeMacro{\catspace}
% \catcodeof\ whitespace characters.
%
% \DescribeMacro{\catletter}
% \catcodeof\ characters that can be used to name macros.
%
% \DescribeMacro{\catother}
% \catcodeof\ other characters.
%
% \DescribeMacro{\catactive}
% \catcodeof\ ``active'' characters, which can be expanded like single-character macros (with no leading \textbackslash needed).
%
% \DescribeMacro{\catcomment}
% \catcodeof\ characters used to begin comments.
%
% \DescribeMacro{\catinvalid}
% \catcodeof\ invalid characters.
%
% \subsection{Commands to Change a Character's catcode}
%
% Change a given character's catcode easily.
%
% \DescribeMacro{\ignorenewline}
% Tell \TeX\ to ignore newlines when processing input.
%
% \DescribeMacro{\normalnewline}
% Tell \TeX\ not to ignore newlines when processing input.
%
% \DescribeMacro{\ignorespace}
% Tell \TeX\ to ignore space when processing input.
%
% \DescribeMacro{\normalspace}
% Tell \TeX\ not to ignore space when processing input.
%
% \DescribeMacro{\makecatescape}
% |\makecatescape|\marg{char}: Change the catcode of \marg{char} to |\catescape|.
%
% \DescribeMacro{\makecatbgroup}
% |\makecatbgroup|\marg{char}: Change the catcode of \marg{char} to |\catbgroup|.
%
%\StopEventually{^^A
%  \break
%  \PrintChanges
%  \break
%  \PrintIndex
%}
%
% \section{Implementation}
%
%    \begin{macrocode}
%<*package>
%    \end{macrocode}
%
%% \subsection{Symbolic Names for Category Codes}
%
% The implementation of these is trivial -- they're simply
% \verb!\chardef!s.
%
%
%\begin{macro}{\catescape}
%    \begin{macrocode}
\chardef\catescape=0
%    \end{macrocode}
%\end{macro}
%
%
% \begin{macro}{\catbgroup}
%    \begin{macrocode}
\chardef\catbgroup=1
%    \end{macrocode}
% \end{macro}
%
%
% \begin{macro}{\categroup}
%    \begin{macrocode}
\chardef\categroup=2
%    \end{macrocode}
% \end{macro}
%
%
% \begin{macro}{\catmathshift}
%    \begin{macrocode}
\chardef\catmathshift=3
%    \end{macrocode}
% \end{macro}
%
%
% \begin{macro}{\catalign}
%    \begin{macrocode}
\chardef\catalign=4
%    \end{macrocode}
% \end{macro}
%
%
% \begin{macro}{\catendline}
%    \begin{macrocode}
\chardef\catendline=5
%    \end{macrocode}
% \end{macro}
%
%
% \begin{macro}{\catparam}
%    \begin{macrocode}
\chardef\catparam=6
%    \end{macrocode}
% \end{macro}
%
%
% \begin{macro}{\catsuper}
%    \begin{macrocode}
\chardef\catsuper=7
%    \end{macrocode}
% \end{macro}
%
%
% \begin{macro}{\catsub}
%    \begin{macrocode}
\chardef\catsub=8
%    \end{macrocode}
% \end{macro}
%
%
% \begin{macro}{\catignore}
%    \begin{macrocode}
\chardef\catignore=9
%    \end{macrocode}
% \end{macro}
%
%
% \begin{macro}{\catspace}
%    \begin{macrocode}
\chardef\catspace=10
%    \end{macrocode}
% \end{macro}
%
%
% \begin{macro}{\catletter}
%    \begin{macrocode}
\chardef\catletter=11
%    \end{macrocode}
% \end{macro}
%
%
% \begin{macro}{\catother}
%    \begin{macrocode}
\chardef\catother=12
%    \end{macrocode}
% \end{macro}
%
%
% \begin{macro}{\catactive}
%    \begin{macrocode}
\chardef\catactive=13
%    \end{macrocode}
% \end{macro}
%
%
% \begin{macro}{\catcomment}
%    \begin{macrocode}
\chardef\catcomment=14
%    \end{macrocode}
% \end{macro}
%
%
% \begin{macro}{\catinvalid}
%    \begin{macrocode}
\chardef\catinvalid=15
%    \end{macrocode}
% \end{macro}
%
%
% \subsection{Commands to Change the Way \TeX\ Handles Whitespace}
%
%
% \begin{macro}{\ignorenewline}
%    \begin{macrocode}
\newcommand{\ignorenewline}{\endlinechar=-1}
%    \end{macrocode}
% \end{macro}
%
%
% \begin{macro}{\normalnewline}
%    \begin{macrocode}
\newcommand{\normalnewline}{\endlinechar=`\^^M}
%    \end{macrocode}
% \end{macro}
%
%
% \begin{macro}{\ignorespace}
%    \begin{macrocode}
\newcommand{\ignorespace}{\makecatignore{\ }}
%    \end{macrocode}
% \end{macro}
%
%
% \begin{macro}{\normalspace}
%    \begin{macrocode}
\newcommand{\normalspace}{\makecatspace{\ }}
%    \end{macrocode}
% \end{macro}
%
%
% \subsection{Commands to Change a Character's catcode}
%
%
% \begin{macro}{\makecatescape}
%    \begin{macrocode}
\newcommand{\makecatescape}[1]{\catcode`#1=\catescape}
%    \end{macrocode}
% \end{macro}
%
%
% \begin{macro}{\makecatbgroup}
%    \begin{macrocode}
\newcommand{\makecatbgroup}[1]{\catcode`#1=\catbgroup}
%    \end{macrocode}
% \end{macro}
%
%
% \begin{macro}{\makecategroup}
%    \begin{macrocode}
\newcommand{\makecategroup}[1]{\catcode`#1=\categroup}
%    \end{macrocode}
% \end{macro}
%
%
% \begin{macro}{\makecatmathshift}
%    \begin{macrocode}
\newcommand{\makecatmathshift}[1]{\catcode`#1 =\catmathshift}
%    \end{macrocode}
% \end{macro}
%
%
% \begin{macro}{\makecatalign}
%    \begin{macrocode}
\newcommand{\makecatalign}[1]{\catcode`#1=\catalign}
%    \end{macrocode}
% \end{macro}
%
%
% \begin{macro}{\makecatendline}
%    \begin{macrocode}
\newcommand{\makecatendline}[1]{\catcode`#1=\catendline}
%    \end{macrocode}
% \end{macro}
%
%
% \begin{macro}{\makecatparam}
%    \begin{macrocode}
\newcommand{\makecatparam}[1]{\catcode`#1=\catparam}
%    \end{macrocode}
% \end{macro}
%
%
% \begin{macro}{\makecatsuper}
%    \begin{macrocode}
\newcommand{\makecatsuper}[1]{\catcode`#1=\catsuper}
%    \end{macrocode}
% \end{macro}
%
%
% \begin{macro}{\makecatsub}
%    \begin{macrocode}
\newcommand{\makecatsub}[1]{\catcode`#1=\catsub}
%    \end{macrocode}
% \end{macro}
%
%
% \begin{macro}{\makecatignore}
%    \begin{macrocode}
\newcommand{\makecatignore}[1]{\catcode`#1=\catignore}
%    \end{macrocode}
% \end{macro}
%
%
% \begin{macro}{\makecatspace}
%    \begin{macrocode}
\newcommand{\makecatspace}[1]{\catcode`#1=\catspace}
%    \end{macrocode}
% \end{macro}
%
%
% \begin{macro}{\makecatletter}
%    \begin{macrocode}
\newcommand{\makecatletter}[1]{\catcode`#1=\catletter}
%    \end{macrocode}
% \end{macro}
%
%
% \begin{macro}{\makecatother}
%    \begin{macrocode}
\newcommand{\makecatother}[1]{\catcode`#1=\catother}
%    \end{macrocode}
% \end{macro}
%
%
% \begin{macro}{\makecatcomment}
%    \begin{macrocode}
\newcommand{\makecatcomment}[1]{\catcode`#1=\catcomment}
%    \end{macrocode}
% \end{macro}
%
%
% \begin{macro}{\makecatactive}
%    \begin{macrocode}
\newcommand{\makecatactive}[1]{\catcode`#1=\catactive}
%    \end{macrocode}
% \end{macro}
%
%
% \begin{macro}{\makecatinvalid}
%    \begin{macrocode}
\newcommand{\makecatinvalid}[1]{\catcode`#1=\catinvalid}
%    \end{macrocode}
% \end{macro}
%
%
%    \begin{macrocode}
\endinput
%</package>
%    \end{macrocode}
%
%
%
% \CheckSum{112}
%
% \CharacterTable
%  {Upper-case    \A\B\C\D\E\F\G\H\I\J\K\L\M\N\O\P\Q\R\S\T\U\V\W\X\Y\Z
%   Lower-case    \a\b\c\d\e\f\g\h\i\j\k\l\m\n\o\p\q\r\s\t\u\v\w\x\y\z
%   Digits        \0\1\2\3\4\5\6\7\8\9
%   Exclamation   \!     Double quote  \"     Hash (number) \#
%   Dollar        \$     Percent       \%     Ampersand     \&
%   Acute accent  \'     Left paren    \(     Right paren   \)
%   Asterisk      \*     Plus          \+     Comma         \,
%   Minus         \-     Point         \.     Solidus       \/
%   Colon         \:     Semicolon     \;     Less than     \<
%   Equals        \=     Greater than  \>     Question mark \?
%   Commercial at \@     Left bracket  \[     Backslash     \\
%   Right bracket \]     Circumflex    \^     Underscore    \_
%   Grave accent  \`     Left brace    \{     Vertical bar  \|
%   Right brace   \}     Tilde         \~}
%
% \Finale
% \endinput
% Local Variables:
% mode: doctex
% TeX-master: t
% End:
