% \iffalse meta-comment
%<*internal>
\iffalse
%</internal>
%<*readme>
----------------------------------------------------------------
package --- Macros for writing packages
Author:  Mike Conley
E-mail:  conasdf@gmail.com
License: Released under the LaTeX Project Public License v1.3c or later
See:     http://www.latex-project.org/lppl.txt
----------------------------------------------------------------
%</readme>
%<*internal>
\fi
\def\nameofplainTeX{plain}
\ifx\fmtname\nameofplainTeX\else
  \expandafter\begingroup
\fi
%</internal>
%<*install>
\input docstrip.tex
\keepsilent
\askforoverwritefalse
\preamble
----------------------------------------------------------------
package --- Macros for writing packages
Author:  Mike Conley
E-mail:  conasdf@gmail.com
License: Released under the LaTeX Project Public License v1.3c or later
See:     http://www.latex-project.org/lppl.txt
----------------------------------------------------------------

\endpreamble
\postamble

Copyright (C) 2015 by Mike Conley <conasdf@gmail.com>

This work may be distributed and/or modified under the
conditions of the LaTeX Project Public License (LPPL), either
version 1.3c of this license or (at your option) any later
version.  The latest version of this license is in the file:

http://www.latex-project.org/lppl.txt

This work is "maintained" (as per LPPL maintenance status) by
Mike Conley.

This work consists of the file package.dtx and a Makefile.
Running "make" generates the derived files README, package.pdf and package.sty.
Running "make inst" installs the files in the user's TeX tree.
Running "make install" installs the files in the local TeX tree.

\endpostamble

\usedir{tex/latex/package}
\generate{
  \file{\jobname.sty}{\from{\jobname.dtx}{package}}
}
%</install>
%<install>\endbatchfile
%<*internal>
\usedir{source/latex/package}
\generate{
  \file{\jobname.ins}{\from{\jobname.dtx}{install}}
}
\nopreamble\nopostamble
\usedir{doc/latex/package}
\generate{
  \file{README-\jobname.txt}{\from{\jobname.dtx}{readme}}
}
\ifx\fmtname\nameofplainTeX
  \expandafter\endbatchfile
\else
  \expandafter\endgroup
\fi
%</internal>
% \fi
%
% \iffalse
%<*driver>
\ProvidesFile{package.dtx}
%</driver>
%<package>\NeedsTeXFormat{LaTeX2e}[1999/12/01]
%<package>\ProvidesPackage{package}
%<package>\RequirePackage{catcode}
%<*package>
    [2015/07/16 v1.00 Macros for writing packages]
%</package>
%<*driver>
\documentclass{ltxdoc}
\usepackage[a4paper,margin=25mm,left=50mm,nohead]{geometry}
\usepackage[numbered]{hypdoc}
\usepackage{\jobname}
\EnableCrossrefs
\CodelineIndex
\RecordChanges
\begin{document}
  \DocInput{\jobname.dtx}
\end{document}
%</driver>
% \fi
%
% \GetFileInfo{\jobname.dtx}
% \DoNotIndex{\def,\let,\newcommand,\newenvironment,\EndPackage,\endinput}

\StopEventually{
   \break
   \PrintChanges
   \break
   \PrintIndex
}

%
%\title{\textsf{package} --- Macros for writing \LaTeX\ packages\thanks{This file
%   describes version \fileversion, last revised \filedate.}
%}
%\author{Mike Conley\thanks{E-mail: conasdf@gmail.com}}
%\date{Released \filedate}
%
%\maketitle
%
%\changes{v1.00}{2015/07/16}{First public release}
%
% \begin{abstract}
%   \textsf{Package} provides a nice pseudo-environment for \LaTeX{}
%   package writers.  Within \textsf{Package}, spaces and newlines are
%   insignificant, so you don't have to put all those annoying \%'s at
%   the end of every line of your code.
% \end{abstract}
%
% \section{Usage}
%
% Here's a sample package, \textsf{sample}, implemented using
% \textsf{package}.  You'll notice that it looks just like an ordinary
% environment, except that it doesn't use \verb!\begin! and
%   \verb!\end!, which are only allowed after \verb!\begin{document}!.
%   Instead, the pseudo-environment is delimited by
%   \verb!\Package{sample}! and \verb!\EndPackage!.
%
% \begin{verbatim}
% \NeedsTeXFormat{LaTeX2e}
% \RequirePackage{package}
% \Package{sample}
%    \newcommand{\samplemacro}{
%      No comments needed,
%
%      and embedded newlines are OK, too!
%    }
% \EndPackage
% \end{verbatim}
%
% If we were to import the \textsf{sample} package, then
% \verb!\samplemacro! would be typeset as:
%
% Nocommentsneeded,andembeddednewlinesareOK,too!
%
% Oops!  Since \TeX\ is ignoring spaces, we got no spaces \emph{at
% all} in the output!  To fix this, we need \verb!\NormalWhitespace!
% and a group, or \verb!\NormalWhitespaceIn!, which handles the group
% for you.  An example of the second option:
% \begin{verbatim}
%  \newcommand{\samplemacro}{
%     \NormalWhitespaceIn{No comments needed, }
%
%     \NormalWhitespaceIn{and embedded newlines are OK, too!}
%  }
% \end{verbatim}
%
% \DescribeMacro{\IgnoreWhitespace}
%
% \verb!\IgnoreWhitespace! tells \TeX\ to ignore spaces and newlines until
% told otherwise.
%
% \DescribeMacro{\NormalWhitespace}
%
% \verb!\NormalWhitespace! tells \TeX\ not to ignore spaces or newlines
% until told otherwise.
%
% \DescribeMacro{\IgnoreWhitespaceIn}
%
% \verb!\IgnoreWhitespaceIn!\marg{text} tells \TeX\ to typeset
% \meta{text}, ignoring spaces and newlines.
%
% \DescribeMacro{\NormalWhitespaceIn}
%
% \verb!\NormalWhitespaceIn!\marg{text} tells \TeX\ to typeset
% \meta{text} without ignoring spaces or newlines.
%
% \DescribeMacro{\NeedsPackage}
%
% A safe version of \verb!\RequirePackage! that ensures spaces and
% newlines are \emph{not} ignored within the loaded package.
%
% \DescribeMacro{\Package}
%
% Starts a package.  \verb!\Package!\oarg{desc}\marg{name} tells
% \LaTeX\ to create a package named \meta{name} with optional
% description \meta{desc}.  Until told otherwise, \TeX\ will ignore
% spaces and newlines.
%
% \DescribeMacro{\EndPackage}
%
% End your package.  This tells \TeX\ to once again consider spaces
% and newlines, and lets it know that nothing further in the text file
% should be processed.
%
%
% \section{Implementation}
%
%    \begin{macrocode}
%<*package>

%    \end{macrocode}
%
% \begin{macro}{\IgnoreWhitespace}
%
% Tell \TeX\ to ignore whitespace using macros from the \texttt{catcode} package.
%    \begin{macrocode}
\newcommand{\IgnoreWhitespace}{%
  \ignorespace%
  \ignorenewline%
}
%    \end{macrocode}
% \end{macro}
%
%
% \begin{macro}{\NormalWhitespace}
%     \begin{macrocode}
\newcommand{\NormalWhitespace}{%
  \normalnewline%
  \normalspace%
}
%    \end{macrocode}
% \end{macro}
%
%
% \begin{macro}{\IgnoreWhitespaceIn}
%    \begin{macrocode}
\newcommand{\IgnoreWhitespaceIn}[1]{{%
  \IgnoreWhitespace%
  #1%
}}
%    \end{macrocode}
% \end{macro}
%
%
% \begin{macro}{\NormalWhitespaceIn}
%    \begin{macrocode}
\newcommand{\NormalWhitespaceIn}[1]{{%
  \NormalWhitespace%
  #1%
}}
%    \end{macrocode}
% \end{macro}
%
%
% \begin{macro}{\NeedsPackage}
%    \begin{macrocode}
\newcommand{\NeedsPackage}[1]{%
  \@ifnextchar[{\package@needspackage{#1}}%
               {\package@needspackage{#1}[]}%] <- keep editors happy with matching pairs
}
\def\package@needspackage#1[#2]{%
  \NormalWhitespace%
  \RequirePackage{#1}[#2]%
  \IgnoreWhitespace%
}
%    \end{macrocode}
% \end{macro}
%
%
% \begin{macro}{\Package}
%
%   This macro changes the category codes for spaces and newlines,
%   making them insignificant to \TeX.  It also declares your package
%   for you.  Note that the default value for the optional argument is
%   [], i.e., no description is given to \verb!\ProvidesPackage! by
%   default.
%   \\
%   This macro was constructed with the aid of \textsf{newcommand.py},
%   available at \verb!ctan:/support/newcommand!.
%
%    \begin{macrocode}
\newcommand{\Package}[1]{%
  \@ifnextchar[{\package@package{#1}}%
               {\package@package{#1}[]}%]
}
\def\package@package#1[#2]{%
  \ProvidesPackage{#1}[#2]%
  \IgnoreWhitespace%
}
%    \end{macrocode}
% \end{macro}
%
% \begin{macro}{\EndPackage}
%
%   Resets the category codes for spaces and endlines, and produces an
%   \verb!\endinput! to finish off your package.
%
%   We have to use \verb!\def!, because \verb!\newcommand! refuses to
%   accept any macro name beginning with \verb!\end!.
%
%    \begin{macrocode}
\def\EndPackage{%
  \NormalWhitespace%
  \endinput%
}
%    \end{macrocode}
% \end{macro}
%
%    \begin{macrocode}
\endinput
%</package>
%    \end{macrocode}
%
%
% \CheckSum{38}
%
% \CharacterTable
%  {Upper-case    \A\B\C\D\E\F\G\H\I\J\K\L\M\N\O\P\Q\R\S\T\U\V\W\X\Y\Z
%   Lower-case    \a\b\c\d\e\f\g\h\i\j\k\l\m\n\o\p\q\r\s\t\u\v\w\x\y\z
%   Digits        \0\1\2\3\4\5\6\7\8\9
%   Exclamation   \!     Double quote  \"     Hash (number) \#
%   Dollar        \$     Percent       \%     Ampersand     \&
%   Acute accent  \'     Left paren    \(     Right paren   \)
%   Asterisk      \*     Plus          \+     Comma         \,
%   Minus         \-     Point         \.     Solidus       \/
%   Colon         \:     Semicolon     \;     Less than     \<
%   Equals        \=     Greater than  \>     Question mark \?
%   Commercial at \@     Left bracket  \[     Backslash     \\
%   Right bracket \]     Circumflex    \^     Underscore    \_
%   Grave accent  \`     Left brace    \{     Vertical bar  \|
%   Right brace   \}     Tilde         \~}
%
%
%\Finale
% \endinput
% Local Variables:
% mode: doctex
% TeX-master: t
% End:
