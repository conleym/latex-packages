% \iffalse meta-comment
%<*internal>
\iffalse
%</internal>
%<*readme>
----------------------------------------------------------------
ifmm --- A Robust ifmmode
Author:  Mike Conley
E-mail:  conasdf@gmail.com
License: Released under the LaTeX Project Public License v1.3c or later
See:     http://www.latex-project.org/lppl.txt
----------------------------------------------------------------
%</readme>
%<*internal>
\fi
\def\nameofplainTeX{plain}
\ifx\fmtname\nameofplainTeX\else
  \expandafter\begingroup
\fi
%</internal>
%<*install>
\input docstrip.tex
\keepsilent
\askforoverwritefalse
\preamble
----------------------------------------------------------------
ifmm --- A Robust ifmmode
Author:  Mike Conley
E-mail:  conasdf@gmail.com
License: Released under the LaTeX Project Public License v1.3c or later
See:     http://www.latex-project.org/lppl.txt
----------------------------------------------------------------

\endpreamble
\postamble

Copyright (C) 2015 by Mike Conley <conasdf@gmail.com>

This work may be distributed and/or modified under the
conditions of the LaTeX Project Public License (LPPL), either
version 1.3c of this license or (at your option) any later
version.  The latest version of this license is in the file:

http://www.latex-project.org/lppl.txt

This work is "maintained" (as per LPPL maintenance status) by
Mike Conley.

This work consists of the file ifmm.dtx and a Makefile.
Running "make" generates the derived files README, ifmm.pdf and ifmm.sty.
Running "make inst" installs the files in the user's TeX tree.
Running "make install" installs the files in the local TeX tree.

\endpostamble

\usedir{tex/latex/ifmm}
\generate{
  \file{\jobname.sty}{\from{\jobname.dtx}{package}}
}
%</install>
%<install>\endbatchfile
%<*internal>
\usedir{source/latex/ifmm}
\generate{
  \file{\jobname.ins}{\from{\jobname.dtx}{install}}
}
\nopreamble\nopostamble
\usedir{doc/latex/ifmm}
\generate{
  \file{README-\jobname.txt}{\from{\jobname.dtx}{readme}}
}
\ifx\fmtname\nameofplainTeX
  \expandafter\endbatchfile
\else
  \expandafter\endgroup
\fi
%</internal>
% \fi
%
% \iffalse
%<*driver>
\ProvidesFile{ifmm.dtx}
%</driver>
%<*package>
\NeedsTeXFormat{LaTeX2e}[1999/12/01]
\RequirePackage{package}
\Package{ifmm}[2015/07/18 v1.00 A Robust ifmmode]
%</package>
%<*driver>
\NormalWhitespace
\documentclass{ltxdoc}
\usepackage[a4paper,margin=25mm,left=50mm,nohead]{geometry}
\usepackage[numbered]{hypdoc}
\usepackage{\jobname}
\begin{document}
  \DocInput{\jobname.dtx}
\end{document}
%</driver>
% \fi
%
%
%\title{\textsf{ifmm} --- A Robust \texttt{\textbackslash{}ifmmode}\thanks{This file
%   describes version \fileversion, last revised \filedate.}
%}
%\author{Mike Conley\thanks{E-mail: conasdf@gmail.com}}
%\date{Released \filedate}
%
%\maketitle
%
%\changes{v1.00}{2015/07/18}{First public release}
%
% \section{Usage}
%
% \DescribeMacro{\ifmm}
% |\ifmm|\marg{mathmode}\marg{textmode}
%
% An if-then-else construct.  Expands to the first argument if
% expanded in math mode, and the second argument otherwise.  Unlike
% the standard \LaTeX\ |\ifmmode|, |\ifmm| is robust.
%
%
% \section{Implementation}
%
%    \begin{macrocode}
%<*package>

%    \end{macrocode}
%
% \begin{macro}{\ifmm}
%    \begin{macrocode}
\DeclareRobustCommand{\ifmm}[2]{\ifmmode #1\else #2\fi}
%    \end{macrocode}
% \end{macro}
%
%    \begin{macrocode}
\EndPackage
%</package>
%    \end{macrocode}
%
% \CheckSum{6}
%
% \CharacterTable
%  {Upper-case    \A\B\C\D\E\F\G\H\I\J\K\L\M\N\O\P\Q\R\S\T\U\V\W\X\Y\Z
%   Lower-case    \a\b\c\d\e\f\g\h\i\j\k\l\m\n\o\p\q\r\s\t\u\v\w\x\y\z
%   Digits        \0\1\2\3\4\5\6\7\8\9
%   Exclamation   \!     Double quote  \"     Hash (number) \#
%   Dollar        \$     Percent       \%     Ampersand     \&
%   Acute accent  \'     Left paren    \(     Right paren   \)
%   Asterisk      \*     Plus          \+     Comma         \,
%   Minus         \-     Point         \.     Solidus       \/
%   Colon         \:     Semicolon     \;     Less than     \<
%   Equals        \=     Greater than  \>     Question mark \?
%   Commercial at \@     Left bracket  \[     Backslash     \\
%   Right bracket \]     Circumflex    \^     Underscore    \_
%   Grave accent  \`     Left brace    \{     Vertical bar  \|
%   Right brace   \}     Tilde         \~}
%
% \endinput
% Local Variables:
% mode: doctex
% TeX-master: t
% End:
