% \iffalse meta-comment
%<*internal>
\iffalse
%</internal>
%<*readme>
----------------------------------------------------------------
mcsymb --- Useful composite symbols
Author:  Mike Conley
E-mail:  conasdf@gmail.com
License: Released under the LaTeX Project Public License v1.3c or later
See:     http://www.latex-project.org/lppl.txt
----------------------------------------------------------------
%</readme>
%<*internal>
\fi
\def\nameofplainTeX{plain}
\ifx\fmtname\nameofplainTeX\else
  \expandafter\begingroup
\fi
%</internal>
%<*install>
\input docstrip.tex
\keepsilent
\askforoverwritefalse
\preamble
----------------------------------------------------------------
mcsymb --- Useful composite symbols
Author:  Mike Conley
E-mail:  conasdf@gmail.com
License: Released under the LaTeX Project Public License v1.3c or later
See:     http://www.latex-project.org/lppl.txt
----------------------------------------------------------------

\endpreamble
\postamble

Copyright (C) 2017 by Mike Conley <conasdf@gmail.com>

This work may be distributed and/or modified under the
conditions of the LaTeX Project Public License (LPPL), either
version 1.3c of this license or (at your option) any later
version.  The latest version of this license is in the file:

http://www.latex-project.org/lppl.txt

This work is "maintained" (as per LPPL maintenance status) by
Mike Conley.

This work consists of the file mcsymb.dtx and a Makefile.
Running "make" generates the derived files README, mcsymb.pdf and mcsymb.sty.
Running "make inst" installs the files in the user's TeX tree.
Running "make install" installs the files in the local TeX tree.

\endpostamble

\usedir{tex/latex/mcsymb}
\generate{
  \file{\jobname.sty}{\from{\jobname.dtx}{package}}
}
%</install>
%<install>\endbatchfile
%<*internal>
\usedir{source/latex/mcsymb}
\generate{
  \file{\jobname.ins}{\from{\jobname.dtx}{install}}
}
\nopreamble\nopostamble
\usedir{doc/latex/mcsymb}
\generate{
  \file{README-\jobname.txt}{\from{\jobname.dtx}{readme}}
}
\ifx\fmtname\nameofplainTeX
  \expandafter\endbatchfile
\else
  \expandafter\endgroup
\fi
%</internal>
% \fi
%
% \iffalse
%<*driver>
\ProvidesFile{mcsymb.dtx}
%</driver>
%<package>\NeedsTeXFormat{LaTeX2e}[1999/12/01]
%<package>\RequirePackage{package, gbox, overstrike, regmath, localloc}
%<package>\Package{mcsymb}
%<*package>
    [2005/08/30 v1.00 Useful composite symbols]
%</package>
%<*driver>
\documentclass{ltxdoc}
\usepackage[a4paper,margin=25mm,left=50mm,nohead]{geometry}
\usepackage[numbered]{hypdoc}
\usepackage{amsmath, fancyvrb}
\usepackage{\jobname}
\begin{document}
  \DocInput{\jobname.dtx}
\end{document}
%</driver>
% \fi
%
%
%\title{\textsf{mcsymb} --- Useful Composite Symbols\thanks{This file
%   describes version \fileversion, last revised \filedate.}
%}
%\author{Mike Conley\thanks{E-mail: conasdf@gmail.com}}
%\date{Released \filedate}
%
%\maketitle
%\DoNotIndex{\mathop, \overstrike}
%
%\changes{v1.00}{2005/08/30}{First public release}
%
% \begin{abstract}

% \end{abstract}
%
% \section{Usage}
%
% \DescribeMacro{\dotcup}
%
% Set union with a dot: $\dotcup$
%
% \DescribeMacro{\bigdotcup}
%
% Big version of |\dotcup|: $\bigdotcup$
%
% \DescribeMacro{\dotcap}
%
% Set intersection with a dot: $\dotcap$
%
% \DescribeMacro{\bigdotcap}
%
% Big version of |\dotcap|: $\bigdotcap$
%
% \DescribeMacro{\turnstyle}
% \DescribeMacro{\rturnstyle}
%
% The turnstyle symbol: $\turnstyle$
%
% Unlike |\vdash| ($\vdash$), this is the same size as |\models|, so math
% using both looks nice.
%
% \DescribeMacro{\lturnstyle}
%
% Reverse |\turnstyle|: $\lturnstyle$
%
% \DescribeMacro{\modelled}
%
% Reverse |\models|: $\modelled$
%
% \DescribeMacro{\twiddle}
%
% \SaveVerb{verbsim}|$\sim A$|
% \SaveVerb{verbtwiddle}|$\twiddle A$|
%
% Unary |\sim|: $\twiddle$. For comparison:
% \begin{align*}
%     \text{\UseVerb{verbsim}}     && \sim A     \\
%     \text{\UseVerb{verbtwiddle}} && \twiddle A
% \end{align*}
%
% \DescribeMacro{\uint}
%
% Upper integral symbol: $\uint_a^b$
%
% \DescribeMacro{\lint}
%
% Lower integral symbol: $\lint_a^b$
%
%
%
% \section{Implementation}
%
%    \begin{macrocode}
%<*package>

%    \end{macrocode}
%
%
%
% \begin{macro}{\dotcup}
% \SaveVerb{verbaccent}|$A \mathop{\mathaccent\cdot\cup} B$|
% \SaveVerb{verbdotcup}|$A \dotcup B|
% \newcommand{\altdotcup}{\mathop{\mathaccent\cdot\cup}}
% I think this looks better than the
% \href{http://mirrors.ctan.org/info/symbols/comprehensive/symbols-a4.pdf}
%      {comprehensive \LaTeX\ symbol list's}
% suggestion (|\mathop{\mathaccent\cdot\cup}|).
% Compare foryourself:
%  \begin{align*}
%      \text{\UseVerb{verbdotcup}} && A & \dotcup B \\
%      \text{\UseVerb{verbaccent}} && A & \altdotcup B
%  \end{align*}
%
%    \begin{macrocode}
\newcommand{\dotcup}{\mathop{\overstrike{\cup}{\cdot}}\nolimits}
%    \end{macrocode}
% \end{macro}
%
% \begin{macro}{\bigdotcup}
%    \begin{macrocode}
\newcommand{\bigdotcup}{\mathop{\overstrike{\bigcup}{\cdot}}}
%    \end{macrocode}
% \end{macro}
%
% \begin{macro}{\dotcap}
%    \begin{macrocode}
\newcommand{\dotcap}{\mathop{\overstrike{\cap}{\cdot}}\nolimits}
%    \end{macrocode}
% \end{macro}
%
% \begin{macro}{\bigdotcap}
%    \begin{macrocode}
\newcommand{\bigdotcap}{\mathop{\overstrike{\bigcap}{\cdot}}}
%    \end{macrocode}
% \end{macro}
%
% \begin{macro}{\turnstyle}
%    \begin{macrocode}
\newcommand{\turnstyle}{\mathrel|\joinrel\relbar}
\newcommand{\rturnstyle}{\turnstyle}
%    \end{macrocode}
% \end{macro}
%
% \begin{macro}{\lturnstyle}
%    \begin{macrocode}
\newcommand{\lturnstyle}{\relbar\joinrel\mathrel|}
%    \end{macrocode}
% \end{macro}
%
% \begin{macro}{\modelled}
%    \begin{macrocode}
\newcommand{\modelled}{=\joinrel\mathrel|}
%    \end{macrocode}
% \end{macro}
%
% \begin{macro}{\twiddle}
%    \begin{macrocode}
\newcommand{\twiddle}{\mathord{\sim}}
%    \end{macrocode}
% \end{macro}
%
% \begin{macro}{\uint}
%    \begin{macrocode}
\newcommand{\uint}{{
  \outermathstyle{
    \lnewbox\intbox
    \lnewdimen\intwd

    \setbox\intbox=\hbox{\m@th$\int$}
    \intwd=\wd\intbox
    \rdiv{\intwd}{2}{\intwd}

     \vbox{
%    \end{macrocode}
%
% Setting |\baselineskip| to 1pt causes \TeX\ to put 1pt of glue
% between the |hrule| and the integral symbol.
%
%    \begin{macrocode}
        \baselineskip=1pt
        \moveright \intwd
        \vbox{\hrule width \intwd}
        \hbox{\m@th$\int$}}
}}}
%    \end{macrocode}
% \end{macro}
%
%
% \begin{macro}{\lint}
%    \begin{macrocode}
\newcommand{\lint}{{
    \outermathstyle{
      \lnewbox\intbox
      \lnewdimen\intwd

      \setbox\intbox=\hbox{\m@th$\int$}
      \intwd=\wd\intbox
      \rdiv{\intwd}{2}{\intwd}

      \vtop{
        \hbox{\m@th$\int$}
        \vskip 1pt
        \hrule width \intwd}
}}}
%    \end{macrocode}
% \end{macro}
%
%
%
%
%
%    \begin{macrocode}
\EndPackage
%</package>
%    \end{macrocode}
%
%
% \CheckSum{0}
%
% \CharacterTable
%  {Upper-case    \A\B\C\D\E\F\G\H\I\J\K\L\M\N\O\P\Q\R\S\T\U\V\W\X\Y\Z
%   Lower-case    \a\b\c\d\e\f\g\h\i\j\k\l\m\n\o\p\q\r\s\t\u\v\w\x\y\z
%   Digits        \0\1\2\3\4\5\6\7\8\9
%   Exclamation   \!     Double quote  \"     Hash (number) \#
%   Dollar        \$     Percent       \%     Ampersand     \&
%   Acute accent  \'     Left paren    \(     Right paren   \)
%   Asterisk      \*     Plus          \+     Comma         \,
%   Minus         \-     Point         \.     Solidus       \/
%   Colon         \:     Semicolon     \;     Less than     \<
%   Equals        \=     Greater than  \>     Question mark \?
%   Commercial at \@     Left bracket  \[     Backslash     \\
%   Right bracket \]     Circumflex    \^     Underscore    \_
%   Grave accent  \`     Left brace    \{     Vertical bar  \|
%   Right brace   \}     Tilde         \~}
%
% \endinput
% Local Variables:
% mode: doctex
% TeX-master: t
% End:
