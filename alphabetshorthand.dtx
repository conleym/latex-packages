% \iffalse meta-comment
%<*internal>
\iffalse
%</internal>
%<*readme>
----------------------------------------------------------------
alphabetshorthand ---  Simple method for declaring shorthand commands
Author:  Mike Conley
E-mail:  conasdf@gmail.com
License: Released under the LaTeX Project Public License v1.3c or later
See:     http://www.latex-project.org/lppl.txt
----------------------------------------------------------------
%</readme>
%<*internal>
\fi
\def\nameofplainTeX{plain}
\ifx\fmtname\nameofplainTeX\else
  \expandafter\begingroup
\fi
%</internal>
%<*install>
\input docstrip.tex
\keepsilent
\askforoverwritefalse
\preamble
----------------------------------------------------------------
alphabetshorthand --- Generic font selection commands
Author:  Mike Conley
E-mail:  conasdf@gmail.com
License: Released under the LaTeX Project Public License v1.3c or later
See:     http://www.latex-project.org/lppl.txt
----------------------------------------------------------------

\endpreamble
\postamble

Copyright (C) 2017 by Mike Conley <conasdf@gmail.com>

This work may be distributed and/or modified under the
conditions of the LaTeX Project Public License (LPPL), either
version 1.3c of this license or (at your option) any later
version.  The latest version of this license is in the file:

http://www.latex-project.org/lppl.txt

This work is "maintained" (as per LPPL maintenance status) by
Mike Conley.

This work consists of the file alphabetshorthand.dtx and a Makefile.
Running "make" generates the derived files README, alphabetshorthand.pdf and alphabetshorthand.sty.
Running "make inst" installs the files in the user's TeX tree.
Running "make install" installs the files in the local TeX tree.

\endpostamble

\usedir{tex/latex/alphabetshorthand}
\generate{
  \file{\jobname.sty}{\from{\jobname.dtx}{package}}
}
%</install>
%<install>\endbatchfile
%<*internal>
\usedir{source/latex/alphabetshorthand}
\generate{
  \file{\jobname.ins}{\from{\jobname.dtx}{install}}
}
\nopreamble\nopostamble
\usedir{doc/latex/alphabetshorthand}
\generate{
  \file{README-\jobname.txt}{\from{\jobname.dtx}{readme}}
}
\ifx\fmtname\nameofplainTeX
  \expandafter\endbatchfile
\else
  \expandafter\endgroup
\fi
%</internal>
% \fi
%
% \iffalse
%<*driver>
\ProvidesFile{alphabetshorthand.dtx}
%</driver>
%<package>\NeedsTeXFormat{LaTeX2e}[1999/12/01]
%<package>\RequirePackage{package, ifthen, regmath}
%<package>\Package{alphabetshorthand}
%<*package>
    [2005/08/30 v1.00 Generic font selection commands]
%</package>
%<*driver>
\documentclass{ltxdoc}
\usepackage[a4paper,margin=25mm,left=50mm,nohead]{geometry}
\usepackage[numbered]{hypdoc}
\usepackage{amsfonts, amstext}
\usepackage{\jobname}
\begin{document}
  \NormalWhitespace
  \DocInput{\jobname.dtx}
\end{document}
%</driver>
% \fi
%
%
% \title{\textsf{alphabetshorthand} --- Declare Shorthand Commands
%  \thanks{This file describes version \fileversion, last revised \filedate.}
%}
%\author{Mike Conley\thanks{E-mail: conasdf@gmail.com}}
%\date{Released \filedate}
%
%\maketitle
%
%\changes{v1.00}{2005/08/30}{First public release}
%
% \begin{abstract}

% \end{abstract}
%
% \section{Usage}
%
% \DescribeMacro{\DeclareAlphabetShorthand}
%
% |\DeclareAlphabetShorthand|\marg{firstletter}\marg{lastletter}\marg{command}\marg{prefix}\marg{suffix}
%
% Declare a series of shorthand commands, one for each letter in the
% range [\meta{firstletter}, \meta{lastletter}]. \meta{firstletter}
% and \meta{lastletter} must be both be ASCII letters, and must both
% be either capital or lowercase. For example, assuming the
% \texttt{amsfonts} package is loaded, one can define commands for
% each blackboard bold letter as follows:
%
% \DeclareAlphabetShorthand{A}{Z}{\mathbb}{s}{bb}
% \[
% |\DeclareAlphabetShorthand{A}{Z}{\mathbb}{s}{bb}|
% \]
% And |$\sAbb, \sBbb, ..., \sZbb$| will result in: $\sAbb, \sBbb, ..., \sZbb$
%
% Both prefix and suffix are optional (empty braces can be given to specify none), as in:
% \[
%   |\DeclareAlphabetShorthand{A}{C}{\mathcal}{}{cal}|,
% \]
% which creates |\Acal|, |\Bcal|, and |\Ccal| and
% \[
% |\DeclareAlphabetShorthand{A}{C}{\mathcal}{cal}{}|,
% \]
% which creates |\calA|, |\calB|, and |\calC|.
%
% \section{Implementation}
%
%    \begin{macrocode}
%<*package>

%    \end{macrocode}
%
%
% \begin{macro}{\DeclareAlphabetShorthand}
%    \begin{macrocode}
\newcommand{\DeclareAlphabetShorthand}[5]{
%    \end{macrocode}
% Everything happens in a group so we don't stomp on anybody's registers.
%    \begin{macrocode}
{
%    \end{macrocode}
%
%    \begin{macrocode}
    \rld{`#1}{\count0}
    \rld{`#2}{\count1}
    \countinc{\count1}

    \def\alphabetshorthand@cmd##1{#3{##1}}
    \def\alphabetshorthand@name##1{#4##1#5}

    \ifthenelse{\count0 < `a \or \count0 > `z}
    {
      \ifthenelse{\count0 < `A \or \count0 > `Z}
                 {\alphabetshorthand@errmsg{#1}{#2}}
                 {
                   \rsub{\count0}{`A}{\count0}
                   \rsub{\count1}{`A}{\count1}
                   \alphabetshorthand@dispatch{\@Alph}
                                              {\alphabetshorthand@cmd}
                                              {\alphabetshorthand@name}
                 }
    }
    {
      \rsub{\count0}{`a}{\count0}
      \rsub{\count1}{`a}{\count1}
      \alphabetshorthand@dispatch{\@alph}
                                 {\alphabetshorthand@cmd}
                                 {\alphabetshorthand@name}
    }
}}
%    \end{macrocode}
% \end{macro}
%
%
% \begin{macro}{\alphabetshorthand@dispatch}
%    \begin{macrocode}
\newcommand{\alphabetshorthand@dispatch}[3]{
  \loop\ifnum\count0 < \count1
    \countinc{\count0}
    \expandafter\xdef \csname#3{#1{\count0}}\endcsname
      { \expandafter \noexpand #2{#1{\count0}} }
  \repeat
}
%    \end{macrocode}
% \end{macro}
%
%
% \begin{macro}{\alphabetshorthand@errmsg}
% Need to switch back to normal whitespace mode to preserve whitespace in the error message.
%    \begin{macrocode}
\NormalWhitespace
\newcommand{\alphabetshorthand@errmsg}[2]
{\errmessage{Some characters in #1 -- #2 are out of range.}}
%    \end{macrocode}
% \end{macro}
%
%
%
%    \begin{macrocode}
\endinput
%</package>
%    \end{macrocode}
%
%
% \CheckSum{63}
%
% \CharacterTable
%  {Upper-case    \A\B\C\D\E\F\G\H\I\J\K\L\M\N\O\P\Q\R\S\T\U\V\W\X\Y\Z
%   Lower-case    \a\b\c\d\e\f\g\h\i\j\k\l\m\n\o\p\q\r\s\t\u\v\w\x\y\z
%   Digits        \0\1\2\3\4\5\6\7\8\9
%   Exclamation   \!     Double quote  \"     Hash (number) \#
%   Dollar        \$     Percent       \%     Ampersand     \&
%   Acute accent  \'     Left paren    \(     Right paren   \)
%   Asterisk      \*     Plus          \+     Comma         \,
%   Minus         \-     Point         \.     Solidus       \/
%   Colon         \:     Semicolon     \;     Less than     \<
%   Equals        \=     Greater than  \>     Question mark \?
%   Commercial at \@     Left bracket  \[     Backslash     \\
%   Right bracket \]     Circumflex    \^     Underscore    \_
%   Grave accent  \`     Left brace    \{     Vertical bar  \|
%   Right brace   \}     Tilde         \~}
%
% \endinput
% Local Variables:
% mode: doctex
% TeX-master: t
% End:
