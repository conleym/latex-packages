% \iffalse meta-comment
%<*internal>
\iffalse
%</internal>
%<*readme>
----------------------------------------------------------------
gcommand --- Define commands that work in both math and text mode.
Author:  Mike Conley
E-mail:  conasdf@gmail.com
License: Released under the LaTeX Project Public License v1.3c or later
See:     http://www.latex-project.org/lppl.txt
----------------------------------------------------------------
%</readme>
%<*internal>
\fi
\def\nameofplainTeX{plain}
\ifx\fmtname\nameofplainTeX\else
  \expandafter\begingroup
\fi
%</internal>
%<*install>
\input docstrip.tex
\keepsilent
\askforoverwritefalse
\preamble
----------------------------------------------------------------
gcommand --- Define commands that work in both math and text mode.
Author:  Mike Conley
E-mail:  conasdf@gmail.com
License: Released under the LaTeX Project Public License v1.3c or later
See:     http://www.latex-project.org/lppl.txt
----------------------------------------------------------------

\endpreamble
\postamble

Copyright (C) 2017 by Mike Conley <conasdf@gmail.com>

This work may be distributed and/or modified under the
conditions of the LaTeX Project Public License (LPPL), either
version 1.3c of this license or (at your option) any later
version.  The latest version of this license is in the file:

http://www.latex-project.org/lppl.txt

This work is "maintained" (as per LPPL maintenance status) by
Mike Conley.

This work consists of the file gcommand.dtx and a Makefile.
Running "make" generates the derived files README, gcommand.pdf and gcommand.sty.
Running "make inst" installs the files in the user's TeX tree.
Running "make install" installs the files in the local TeX tree.

\endpostamble

\usedir{tex/latex/gcommand}
\generate{
  \file{\jobname.sty}{\from{\jobname.dtx}{package}}
}
%</install>
%<install>\endbatchfile
%<*internal>
\usedir{source/latex/gcommand}
\generate{
  \file{\jobname.ins}{\from{\jobname.dtx}{install}}
}
\nopreamble\nopostamble
\usedir{doc/latex/gcommand}
\generate{
  \file{README-\jobname.txt}{\from{\jobname.dtx}{readme}}
}
\ifx\fmtname\nameofplainTeX
  \expandafter\endbatchfile
\else
  \expandafter\endgroup
\fi
%</internal>
% \fi
%
% \iffalse
%<*driver>
\ProvidesFile{gcommand.dtx}
%</driver>
%<package>\NeedsTeXFormat{LaTeX2e}[1999/12/01]
%<package>\RequirePackage{package, ifmm, defcommand}
%<package>\ProvidesPackage{gcommand}
%<*package>
    [2017/06/10 v1.00 Define commands that work in both math and text mode.]
%</package>
%<*driver>
\documentclass{ltxdoc}
\usepackage[a4paper,margin=25mm,left=50mm,nohead]{geometry}
\usepackage[numbered]{hypdoc}
\usepackage{\jobname}
\begin{document}
  \DocInput{\jobname.dtx}
\end{document}
%</driver>
% \fi
%
% \GetFileInfo{\jobname.dtx}
%
%\title{\textsf{gcommand} --- Define commands that work in both math and text mode.\thanks{This file
%   describes version \fileversion, last revised \filedate.}
%}
%\author{Mike Conley\thanks{E-mail: conasdf@gmail.com}}
%\date{Released \filedate}
%
%\maketitle
%
%\changes{v1.00}{2017/06/10}{First public release}
%
% \section{Usage}
%
% \DescribeMacro{\DeclareGenericCommand}
%
% Declare a command that will work in both text and math mode.
%
% \verb!\DeclareGenericCommand!\marg{name}\oarg{nargs}\marg{math}\marg{text}
% declares a macro named \meta{name}, optionally taking \meta{nargs} arguments.
% \verb!\name! will expand to \meta{text} in text mode and \meta{math} in math
% mode.
%
% \DeclareGenericCommand{\example}{a}{b}
% For example, if we declare \verb!\gt! as \verb!\DeclareGenericCommand{\example}{a}{b}!,
% then \verb!\example! typesets as \example, and \verb!$\example$! as $\example$.
%
% \DescribeMacro{\gpalette}
%
% A generic version of \verb!mathpallette!.
%
%
% \section{Implementation}
%
%    \begin{macrocode}
%<*package>
%    \end{macrocode}
%
% \begin{macro}{\DeclareGenericCommand}
%
% A wrapper around \verb!\gcommand@i! that checks for \meta{nargs}.
%    \begin{macrocode}
\def\DeclareGenericCommand#1{
  \@ifnextchar[{\gcommand@i{#1}}
               {\gcommand@i{#1}[0]}%]
}
%    \end{macrocode}
% \end{macro}
%
%
% \begin{macro}{\gcommand@i}
%    \begin{macrocode}
\def\gcommand@i#1[#2]#3#4{
  \defcommand{#1}[#2]{\ifmm{#3}{#4}}
}
%    \end{macrocode}
% \end{macro}
%
%
% \begin{macro}{\gpalette}
%    \begin{macrocode}
\DeclareGenericCommand{\gpalette}[2]{
  \mathpalette#1{#2}}{#1{{}}{#2}
}
%    \end{macrocode}
% \end{macro}
%
%    \begin{macrocode}
\EndPackage
%</package>
%    \end{macrocode}
%
% \CheckSum{13}
%
% \CharacterTable
%  {Upper-case    \A\B\C\D\E\F\G\H\I\J\K\L\M\N\O\P\Q\R\S\T\U\V\W\X\Y\Z
%   Lower-case    \a\b\c\d\e\f\g\h\i\j\k\l\m\n\o\p\q\r\s\t\u\v\w\x\y\z
%   Digits        \0\1\2\3\4\5\6\7\8\9
%   Exclamation   \!     Double quote  \"     Hash (number) \#
%   Dollar        \$     Percent       \%     Ampersand     \&
%   Acute accent  \'     Left paren    \(     Right paren   \)
%   Asterisk      \*     Plus          \+     Comma         \,
%   Minus         \-     Point         \.     Solidus       \/
%   Colon         \:     Semicolon     \;     Less than     \<
%   Equals        \=     Greater than  \>     Question mark \?
%   Commercial at \@     Left bracket  \[     Backslash     \\
%   Right bracket \]     Circumflex    \^     Underscore    \_
%   Grave accent  \`     Left brace    \{     Vertical bar  \|
%   Right brace   \}     Tilde         \~}
%
% \endinput
% Local Variables:
% mode: doctex
% TeX-master: t
% End:
