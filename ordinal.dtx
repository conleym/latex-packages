% \iffalse meta-comment
%<*internal>
\iffalse
%</internal>
%<*readme>
----------------------------------------------------------------
ordinal --- Macros for typesetting ordinal numbers
Author:  Mike Conley
E-mail:  conasdf@gmail.com
License: Released under the LaTeX Project Public License v1.3c or later
See:     http://www.latex-project.org/lppl.txt
----------------------------------------------------------------

Some text about the package: probably the same as the abstract.
%</readme>
%<*internal>
\fi
\def\nameofplainTeX{plain}
\ifx\fmtname\nameofplainTeX\else
  \expandafter\begingroup
\fi
%</internal>
%<*install>
\input docstrip.tex
\keepsilent
\askforoverwritefalse
\preamble
----------------------------------------------------------------
ordinal --- Macros for typesetting ordinal numbers
Author:  Mike Conley
E-mail:  conasdf@gmail.com
License: Released under the LaTeX Project Public License v1.3c or later
See:     http://www.latex-project.org/lppl.txt
----------------------------------------------------------------

\endpreamble
\postamble

Copyright (C) 2005 by Mike Conley <conasdf@gmail.com>

This work may be distributed and/or modified under the
conditions of the LaTeX Project Public License (LPPL), either
version 1.3c of this license or (at your option) any later
version.  The latest version of this license is in the file:

http://www.latex-project.org/lppl.txt

This work is "maintained" (as per LPPL maintenance status) by
Mike Conley.

This work consists of the file ordinal.dtx and a Makefile.
Running "make" generates the derived files README, ordinal.pdf and ordinal.sty.
Running "make inst" installs the files in the user's TeX tree.
Running "make install" installs the files in the local TeX tree.

\endpostamble

\usedir{tex/latex/ordinal}
\generate{
  \file{\jobname.sty}{\from{\jobname.dtx}{package}}
}
%</install>
%<install>\endbatchfile
%<*internal>
\usedir{source/latex/ordinal}
\generate{
  \file{\jobname.ins}{\from{\jobname.dtx}{install}}
}
\nopreamble\nopostamble
\usedir{doc/latex/ordinal}
\generate{
  \file{README-\jobname.txt}{\from{\jobname.dtx}{readme}}
}
\ifx\fmtname\nameofplainTeX
  \expandafter\endbatchfile
\else
  \expandafter\endgroup
\fi
%</internal>
% \fi
%
% \iffalse
%<*driver>
\ProvidesFile{ordinal.dtx}
%</driver>
%<package>\NeedsTeXFormat{LaTeX2e}[1999/12/01]
%<package>\RequirePackage{package, regmath, gscript}
%<package>\Package{ordinal}
%<*package>
    [2005/08/30 v1.00 Macros for typesetting ordinal numbers]
%</package>
%<*driver>
\documentclass{ltxdoc}
\usepackage[a4paper,margin=25mm,left=50mm,nohead]{geometry}
\usepackage[numbered]{hypdoc}
\usepackage{\jobname}
\EnableCrossrefs
\CodelineIndex
\RecordChanges
\begin{document}
  \DocInput{\jobname.dtx}
\end{document}
%</driver>
% \fi
%
% \GetFileInfo{\jobname.dtx}
%
% \DoNotIndex{\badfalse, \badtrue, \count}
%
%\title{\textsf{ordinal} --- Macros for typesetting ordinal numbers\thanks{This file
%   describes version \fileversion, last revised \filedate.}
%}
%\author{Mike Conley\thanks{E-mail: conasdf@gmail.com}}
%\date{Released \filedate}
%
%\maketitle
%
%\changes{v1.00}{2005/08/30}{First public release}
% \section{Usage}
%
% \DescribeMacro{\ordinalsuffixof}
%
% |\ordinalsuffixof|\marg{N} produces the English ordinal suffix appropriate to the number N. For example,
% |\ordinalsuffixof{23}| produces \ordinalsuffixof{23}.
%
% \DescribeMacro{\ordinal}
% |\ordinal|\marg{N} produces the number N, followed by the appropriate English ordinal suffix, typeset in
% superscript. For example, |\ordinal{23}| produces \ordinal{23}.
%
% \section{Implementation}
%
%    \begin{macrocode}
%<*package>

%    \end{macrocode}
%
%
% \begin{macro}{\ordinalsuffixof}
%   Ordinal numbers in English follow a pattern: for most, the
%   suffix depends entirely on the ones digit.  1 - st, 2 - nd, 3 -
%   rd, and, for any other digit - th.  But there is an exception:
%   numbers having 11 - 13 for their two least-significant digits
%   always take the suffix "th".
%
%    \begin{macrocode}
\newcommand{\ordinalsuffixof}[1]{{
%    \end{macrocode}
%
% First, we need a new switch.  This will tell us if the number's last
% two digits are in (10, 14).
%
%    \begin{macrocode}
    \newif\ifbad
    \badfalse
%    \end{macrocode}
%
% Next, compute \#1 mod 100.
%
%    \begin{macrocode}
    \count0 = #1
    \rmod{\count0}{100}{\count1}
%    \end{macrocode}
%
% Check last two digits.
%
%    \begin{macrocode}
    \ifnum \count1 < 14
      \ifnum \count1 > 10
        \relax \badtrue
      \fi
    \fi
%    \end{macrocode}
%
% If last two digits were in (10,14), there's nothing else to do --
% the result is 'th'.
%
%    \begin{macrocode}
   \ifbad th \else
%    \end{macrocode}
%
% Otherwise, compute \#1 mod 10 and go thru the cases.
%
%    \begin{macrocode}
     \count0 = #1
     \rmod{\count0}{10}{\count1}
      \ifcase \count1 th \or st \or nd \or rd \else th \fi
   \fi
}}
%    \end{macrocode}
% \end{macro}
%
%
% \begin{macro}{\ordinal}
%   \begin{macrocode}
\newcommand{\ordinal}[1]{
  #1 \superscript{\ordinalsuffixof{#1}}
}
%    \end{macrocode}
% \end{macro}
%
%
%    \begin{macrocode}
\EndPackage
%</package>
%    \end{macrocode}
%
% \CheckSum{36}
%
% \CharacterTable
%  {Upper-case    \A\B\C\D\E\F\G\H\I\J\K\L\M\N\O\P\Q\R\S\T\U\V\W\X\Y\Z
%   Lower-case    \a\b\c\d\e\f\g\h\i\j\k\l\m\n\o\p\q\r\s\t\u\v\w\x\y\z
%   Digits        \0\1\2\3\4\5\6\7\8\9
%   Exclamation   \!     Double quote  \"     Hash (number) \#
%   Dollar        \$     Percent       \%     Ampersand     \&
%   Acute accent  \'     Left paren    \(     Right paren   \)
%   Asterisk      \*     Plus          \+     Comma         \,
%   Minus         \-     Point         \.     Solidus       \/
%   Colon         \:     Semicolon     \;     Less than     \<
%   Equals        \=     Greater than  \>     Question mark \?
%   Commercial at \@     Left bracket  \[     Backslash     \\
%   Right bracket \]     Circumflex    \^     Underscore    \_
%   Grave accent  \`     Left brace    \{     Vertical bar  \|
%   Right brace   \}     Tilde         \~}
%
% \endinput
% Local Variables:
% mode: doctex
% TeX-master: t
% End:
