% \iffalse meta-comment
%<*internal>
\iffalse
%</internal>
%<*readme>
----------------------------------------------------------------
overstrike --- Typeset some text on top of some other text
Author:  Mike Conley
E-mail:  conasdf@gmail.com
License: Released under the LaTeX Project Public License v1.3c or later
See:     http://www.latex-project.org/lppl.txt
----------------------------------------------------------------
%</readme>
%<*internal>
\fi
\def\nameofplainTeX{plain}
\ifx\fmtname\nameofplainTeX\else
  \expandafter\begingroup
\fi
%</internal>
%<*install>
\input docstrip.tex
\keepsilent
\askforoverwritefalse
\preamble
----------------------------------------------------------------
overstrike --- Typeset some text on top of some other text
Author:  Mike Conley
E-mail:  conasdf@gmail.com
License: Released under the LaTeX Project Public License v1.3c or later
See:     http://www.latex-project.org/lppl.txt
----------------------------------------------------------------

\endpreamble
\postamble

Copyright (C) 2017 by Mike Conley <conasdf@gmail.com>

This work may be distributed and/or modified under the
conditions of the LaTeX Project Public License (LPPL), either
version 1.3c of this license or (at your option) any later
version.  The latest version of this license is in the file:

http://www.latex-project.org/lppl.txt

This work is "maintained" (as per LPPL maintenance status) by
Mike Conley.

This work consists of the file overstrike.dtx and a Makefile.
Running "make" generates the derived files README, overstrike.pdf and overstrike.sty.
Running "make inst" installs the files in the user's TeX tree.
Running "make install" installs the files in the local TeX tree.

\endpostamble

\usedir{tex/latex/overstrike}
\generate{
  \file{\jobname.sty}{\from{\jobname.dtx}{package}}
}
%</install>
%<install>\endbatchfile
%<*internal>
\usedir{source/latex/overstrike}
\generate{
  \file{\jobname.ins}{\from{\jobname.dtx}{install}}
}
\nopreamble\nopostamble
\usedir{doc/latex/overstrike}
\generate{
  \file{README-\jobname.txt}{\from{\jobname.dtx}{readme}}
}
\ifx\fmtname\nameofplainTeX
  \expandafter\endbatchfile
\else
  \expandafter\endgroup
\fi
%</internal>
% \fi
%
% \iffalse
%<*driver>
\ProvidesFile{overstrike.dtx}
%</driver>
%<*package>
\NeedsTeXFormat{LaTeX2e}[1999/12/01]
\RequirePackage{package, gcommand}
\Package{overstrike}
    [2005/08/30 v1.00 Typeset some text on top of some other text]
%</package>
%<*driver>
\NormalWhitespace
\documentclass{ltxdoc}
\usepackage[a4paper,margin=25mm,left=50mm,nohead]{geometry}
\usepackage[numbered]{hypdoc}
\usepackage{\jobname}
\begin{document}
  \DocInput{\jobname.dtx}
\end{document}
%</driver>
% \fi
%
%
%\title{\textsf{overstrike} --- Typeset some text on top of some other text\thanks{This file
%   describes version \fileversion, last revised \filedate.}
%}
%\author{Mike Conley\thanks{E-mail: conasdf@gmail.com}}
%\date{Released \filedate}
%
%\maketitle
%
%\changes{v1.00}{2005/08/30}{First public release}
%
% \begin{abstract}
% This package is based on \textsf{overlay.sty} written by Jeremy
% Gibbons.  The original macros are described in
% ``Hey --- it works!'',
% \href{http://mirrors.ctan.org/info/digests/ttn/ttn3n2.600.ps}{
% \textit{\TeX\ and TUG NEWS Vol. 3, No. 2, 1994}}.
%
% The implementation here, however, is significantly different.  The
% following changes have been made:
% \begin{itemize}
% \item The name --- I think \textit{overstrike} is more appropriate
%   than \textit{overlay} (my understanding is that overstrike is, in
%   fact, the correct technical term).
%
% \item These macros work in both text and math modes.
%
% \item The |\overlaystyle| macro is not defined.
% \end{itemize}
% \end{abstract}
%
% \section{Usage}
%
% \DescribeMacro{\loverstrike}
%
% |\loverstrike|\marg{text1}\marg{text2} typesets \meta{text2}
% argument directly on top of \meta{text1}.  This macro starts with
% ``l" because it works in a very simple way.  It simply typesets
% \meta{text1}, then goes back to the place where it began and
% typesets \meta{text2}.
%
% So, the analogy is this: the first argument is like a box, and the
% second argument is left justified inside that box.
%
% \DescribeMacro{\overstrike}
%
% |\overstrike|\marg{text1}\marg{text2} is just like
% |\roverstrike|, but the second argument is centered over the
% first.
%
% \DescribeMacro{\roverstrike}
%
% |\roverstrike|\marg{text1}\marg{text2} is just like
% |\loverstrike|, but the second argument is right justified
% over the first.
%
%
% \section{Implementation}
%
%    \begin{macrocode}
%<*package>

%    \end{macrocode}
%
%    \begin{macrocode}
\newcommand{\overstrike@mathbox}[2]{\m@th$#1#2$}
\newcommand{\overstrike@textbox}[2]{#2}
%    \end{macrocode}
%
%
% \begin{macro}{\overstrike}
%    \begin{macrocode}
\DeclareGenericCommand{\overstrike}[2]
{{
    \let\overstrike@boxcmd=\overstrike@mathbox
    \overstrike@overstrikeimpl{#1}{#2}
}}
{{
    \let\overstrike@boxcmd=\overstrike@textbox
    \overstrike@overstrikeimpl{#1}{#2}
}}
%    \end{macrocode}
% \end{macro}
%
%
% \begin{macro}{\overstrike@overstrikeimpl}
%    \begin{macrocode}
\newcommand{\overstrike@overstrikeimpl}[2]{
  \gpalette\overstrike@doit{{#1}{#2}{\hfil}{\hfil}}
}
%    \end{macrocode}
% \end{macro}
%
%
% \begin{macro}{\loverstrike}
%    \begin{macrocode}
\DeclareGenericCommand{\loverstrike}[2]
{{
    \let\overstrike@boxcmd=\overstrike@mathbox
    \overstrike@loverstrikeimpl{#1}{#2}
}}
{{
    \let\overstrike@boxcmd=\overstrike@textbox
    \overstrike@loverstrikeimpl{#1}{#2}
}}
%    \end{macrocode}
% \end{macro}
%
%
% \begin{macro}{\overstrike@loverstrikeimpl}
%    \begin{macrocode}
\newcommand{\overstrike@loverstrikeimpl}[2]{
  \gpalette\overstrike@doit{{#1}{#2}{}{\hfil}}
}
%    \end{macrocode}
% \end{macro}
%
%
% \begin{macro}{\roverstrike}
%    \begin{macrocode}
\DeclareGenericCommand{\roverstrike}[2]
{{
    \let\overstrike@boxcmd=\overstrike@mathbox
    \overstrike@roverstrikeimpl{#1}{#2}
}}
{{
    \let\overstrike@boxcmd=\overstrike@textbox
    \overstrike@roverstrikeimpl{#1}{#2}
}}
%    \end{macrocode}
% \end{macro}
%
%
% \begin{macro}{\overstrike@roverstrikeimpl}
%    \begin{macrocode}
\newcommand{\overstrike@roverstrikeimpl}[2]{
  \gpalette\overstrike@doit{{#1}{#2}{\hfil}{}}
}
%    \end{macrocode}
% \end{macro}
%
%
% \begin{macro}{\overstrike@doit}
%
%   This macro's sole purpose in life is to strip the extra braces
%   from the second argument.
%
%    \begin{macrocode}
\newcommand{\overstrike@doit}[2]{\overstrike@@doit#1#2}
%    \end{macrocode}
% \end{macro}
%
%
% \begin{macro}{\overstrike@@doit}
%    \begin{macrocode}
\newcommand{\overstrike@@doit}[5]{
  \setbox0 = \hbox{\overstrike@boxcmd{#1}{#2}}
  \setbox1 = \hbox{\overstrike@boxcmd{#1}{#3}}

  \ifdim \wd0 < \wd1
    \setbox2 = \box1
    \setbox1 = \box0
    \setbox0 = \box2
  \fi

  \hbox{\rlap{\hbox to \wd0 {#4\box1\relax#5}}\box0}
}
%    \end{macrocode}
% \end{macro}
%
%
%    \begin{macrocode}
\EndPackage
%</package>
%    \end{macrocode}
%
%
%
% \CheckSum{80}
%
% \CharacterTable
%  {Upper-case    \A\B\C\D\E\F\G\H\I\J\K\L\M\N\O\P\Q\R\S\T\U\V\W\X\Y\Z
%   Lower-case    \a\b\c\d\e\f\g\h\i\j\k\l\m\n\o\p\q\r\s\t\u\v\w\x\y\z
%   Digits        \0\1\2\3\4\5\6\7\8\9
%   Exclamation   \!     Double quote  \"     Hash (number) \#
%   Dollar        \$     Percent       \%     Ampersand     \&
%   Acute accent  \'     Left paren    \(     Right paren   \)
%   Asterisk      \*     Plus          \+     Comma         \,
%   Minus         \-     Point         \.     Solidus       \/
%   Colon         \:     Semicolon     \;     Less than     \<
%   Equals        \=     Greater than  \>     Question mark \?
%   Commercial at \@     Left bracket  \[     Backslash     \\
%   Right bracket \]     Circumflex    \^     Underscore    \_
%   Grave accent  \`     Left brace    \{     Vertical bar  \|
%   Right brace   \}     Tilde         \~}
%
% \endinput
% Local Variables:
% mode: doctex
% TeX-master: t
% End:
