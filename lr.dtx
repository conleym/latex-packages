% \iffalse meta-comment
%<*internal>
\iffalse
%</internal>
%<*readme>
----------------------------------------------------------------
lr --- Deliminter-related macros
Author:  Mike Conley
E-mail:  conasdf@gmail.com
License: Released under the LaTeX Project Public License v1.3c or later
See:     http://www.latex-project.org/lppl.txt
----------------------------------------------------------------

Some text about the package: probably the same as the abstract.
%</readme>
%<*internal>
\fi
\def\nameofplainTeX{plain}
\ifx\fmtname\nameofplainTeX\else
  \expandafter\begingroup
\fi
%</internal>
%<*install>
\input docstrip.tex
\keepsilent
\askforoverwritefalse
\preamble
----------------------------------------------------------------
lr --- Deliminter-related macros
Author:  Mike Conley
E-mail:  conasdf@gmail.com
License: Released under the LaTeX Project Public License v1.3c or later
See:     http://www.latex-project.org/lppl.txt
----------------------------------------------------------------

\endpreamble
\postamble

Copyright (C) 2005 by Mike Conley <conasdf@gmail.com>

This work may be distributed and/or modified under the
conditions of the LaTeX Project Public License (LPPL), either
version 1.3c of this license or (at your option) any later
version.  The latest version of this license is in the file:

http://www.latex-project.org/lppl.txt

This work is "maintained" (as per LPPL maintenance status) by
Mike Conley.

This work consists of the file lr.dtx and a Makefile.
Running "make" generates the derived files README, lr.pdf and lr.sty.
Running "make inst" installs the files in the user's TeX tree.
Running "make install" installs the files in the local TeX tree.

\endpostamble

\usedir{tex/latex/lr}
\generate{
  \file{\jobname.sty}{\from{\jobname.dtx}{package}}
}
%</install>
%<install>\endbatchfile
%<*internal>
\usedir{source/latex/lr}
\generate{
  \file{\jobname.ins}{\from{\jobname.dtx}{install}}
}
\nopreamble\nopostamble
\usedir{doc/latex/lr}
\generate{
  \file{README-\jobname.txt}{\from{\jobname.dtx}{readme}}
}
\ifx\fmtname\nameofplainTeX
  \expandafter\endbatchfile
\else
  \expandafter\endgroup
\fi
%</internal>
% \fi
%
% \iffalse
%<*driver>
\ProvidesFile{lr.dtx}
%</driver>
%<package>\NeedsTeXFormat{LaTeX2e}[1999/12/01]
%<package>\RequirePackage{package}
%<package>\Package{lr}
%<*package>
    [2005/08/30 v1.00 Deliminter-related macros]
%</package>
%<*driver>
\documentclass{ltxdoc}
\usepackage[a4paper,margin=25mm,left=50mm,nohead]{geometry}
\usepackage[numbered]{hypdoc}
\usepackage{\jobname}
\EnableCrossrefs
\CodelineIndex
\RecordChanges
\begin{document}
  \DocInput{\jobname.dtx}
\end{document}
%</driver>
% \fi
%
% \GetFileInfo{\jobname.dtx}
% \DoNotIndex{\newcommand,\newenvironment}
%
%\title{\textsf{lr} --- Deliminter-related macros\thanks{This file
%   describes version \fileversion, last revised \filedate.}
%}
%\author{Mike Conley\thanks{E-mail: conasdf@gmail.com}}
%\date{Released \filedate}
%
%\maketitle
%
%\changes{v1.00}{2005/08/30}{First public release}
%
% \begin{abstract}
%   The \textsf{lr} package contains macros designed to make dealing with
%   delimiters simple.  First, macros are provided to automate size matching,
%   so you don't have to worry about matching |\Bigr| with |\Bigl| or |\right|
%   with |\left|, for example.
%
%   Also available are macros that automatically match left and right
%   delimiters (e.g., |\langle| and |\rangle|).
%
%   This is very similar to (and probably inferior to)
%   |\DeclarePairedDelimiter| from the \textsf{mathtools} package.
%
% \end{abstract}
%
% \section{Usage}
%
%
% \section{Implementation}
%
%    \begin{macrocode}
%<*package>

%    \end{macrocode}
%
% \subsection{Supplementary Delimiter Definitions}
%
% We need to define these so we can use all available delimiters with
% |\lrpair|.
%
% \begin{macro}{\lslash}
%    \begin{macrocode}
\newcommand{\lslash}{/}
%    \end{macrocode}
% \end{macro}
%
% \begin{macro}{\rslash}
%    \begin{macrocode}
\newcommand{\rslash}{\backslash}
%    \end{macrocode}
% \end{macro}
%
% \begin{macro}{\lparen}
%    \begin{macrocode}
\newcommand{\lparen}{(}
%    \end{macrocode}
% \end{macro}
%
% \begin{macro}{\lparen}
%    \begin{macrocode}
\newcommand{\rparen}{)}
%    \end{macrocode}
% \end{macro}
%
% \subsection{Macros for Delimiting Mathematical Formulae}
%
% \begin{macro}{\lr@ifdefault}
%   This just checks to see if \#1 is one of the ``default'' sizes:
%   left, middle, or right.  If any of these are given, it expands to
%   \#2.  Otherwise, expands to \#3.
%
%    \begin{macrocode}
\newcommand{\lr@ifdefault}[3]{
    \newif\iflr@check

    \def\lr@one{#1}
    \def\lr@L{left}
    \def\lr@M{middle}
    \def\lr@R{right}

    \ifx\lr@one\lr@R \lr@checktrue \else
    \ifx\lr@one\lr@L \lr@checktrue \else
    \ifx\lr@one\lr@M \lr@checktrue \fi \fi \fi
%    \end{macrocode}
%
% |\relax| is necessary here --- without it, various macros give mysterious
% errors inside the align environment (and probably other places, too).
%
%    \begin{macrocode}
    \iflr@check\relax
       \def\lr@next{#2}
    \else
       \def\lr@next{#3}
    \fi

    \lr@next
}
%    \end{macrocode}
% \end{macro}
%
%
% \begin{macro}{\lr@lrsize}
%   Sets up |\lr@right| and |\lr@left|.
%
%    \begin{macrocode}
\newcommand{\lr@lrsize}[1]{
  \lr@ifdefault{#1}
               {
                 \let\lr@right=\right
                 \let\lr@left=\left
               }
               {
                 \def\lr@right{\csname #1r \endcsname}
                 \def\lr@left{\csname #1l \endcsname}
               }
}
%    \end{macrocode}
% \end{macro}
%
%
% \begin{macro}{\lr@lrmsize}
% Sets up |\lr@middle|, then calls |\lr@lrsize| to set up |\lr@left| and
% |\lr@right|.
%
%    \begin{macrocode}
\newcommand{\lr@lrmsize}[1]{
  \lr@ifdefault{#1}
               {
                 \let\lr@middle=\middle
                 \let\lr@left=\left
                 \let\lr@right=\right
               }
               {
                 \def\lr@middle{\csname #1m \endcsname}
                 \lr@lrsize{#1}
               }
}
%    \end{macrocode}
% \end{macro}
%
%
% \begin{macro}{\lr@impl}
%    \begin{macrocode}
\newcommand{\lr@impl}[4]{
  \@ifnextchar[{#4{#1}{#2}{#3}}{#4{#1}{#2}{#3}[#2]}%]
}
%    \end{macrocode}
% \end{macro}
%
%
%
% \begin{macro}{\lr@lrpair}
%    \begin{macrocode}
\def\lr@lrpair#1#2#3[#4]{{
  \edef\lr@ldelim{\csname l#2 \endcsname}
  \edef\lr@rdelim{\csname r#4 \endcsname}
  \lr@delimit{#1}{\lr@ldelim}{#3}[\lr@rdelim]
}}
%    \end{macrocode}
% \end{macro}
%
%
% \begin{macro}{\lr@delimit}
%    \begin{macrocode}
\def\lr@delimit#1#2#3[#4]{{
  \lr@lrsize{#1}
  \lr@left #2
  #3
  \lr@right #4
}}
%    \end{macrocode}
% \end{macro}
%
%
% \begin{macro}{\lrpair}
%    \begin{macrocode}
\newcommand{\lrpair}[3][right]{
  \lr@impl{#1}{#2}{#3}{\lr@lrpair}
}
%    \end{macrocode}
% \end{macro}
%
%
% \begin{macro}{\delimit}
%    \begin{macrocode}
\newcommand{\delimit}[3][right]{
  \lr@impl{#1}{#2}{#3}{\lr@delimit}
}
%    \end{macrocode}
% \end{macro}
%
%
% \subsection{Shortcuts for Common Delimiters}
%
% \begin{macro}{\abs}
%    \begin{macrocode}
\newcommand{\abs}[2][right]{\lrpair[#1]{vert}{#2}}
%    \end{macrocode}
% \end{macro}
%
% \begin{macro}{\norm}
%    \begin{macrocode}
\newcommand{\norm}[2][right]{\lrpair[#1]{Vert}{#2}}
%    \end{macrocode}
% \end{macro}
%
% \begin{macro}{\set}
%    \begin{macrocode}
\newcommand{\set}[2][right]{\lrpair[#1]{brace}{#2}}
%    \end{macrocode}
% \end{macro}
%
% \begin{macro}{\ceiling}
%    \begin{macrocode}
\newcommand{\ceiling}[2][right]{\lrpair[#1]{ceil}{#2}}
%    \end{macrocode}
% \end{macro}
%
% \begin{macro}{\floor}
%    \begin{macrocode}
\newcommand{\floor}[2][right]{\lrpair[#1]{floor}{#2}}
%    \end{macrocode}
% \end{macro}
%
%
% \begin{macro}{\parens}
%    \begin{macrocode}
\newcommand{\parens}[2][right]{\lrpair[#1]{paren}{#2}}
%    \end{macrocode}
% \end{macro}
%
% \begin{macro}{\brackets}
%    \begin{macrocode}
\newcommand{\brackets}[2][right]{\lrpair[#1]{brack}{#2}}
%    \end{macrocode}
% \end{macro}
%
% \begin{macro}{\braces}
%    \begin{macrocode}
\newcommand{\braces}[2][right]{\lrpair[#1]{brace}{#2}}
%    \end{macrocode}
% \end{macro}
%
% \begin{macro}{\angles}
%    \begin{macrocode}
\newcommand{\angles}[2][right]{\lrpair[#1]{angle}{#2}}
%    \end{macrocode}
% \end{macro}
%
% \begin{macro}{\downarrows}
%    \begin{macrocode}
\newcommand{\downarrows}[2][right]{\delimit[#1]{\downarrow}{#2}}
%    \end{macrocode}
% \end{macro}
%
% \begin{macro}{\Downarrows}
%    \begin{macrocode}
\newcommand{\Downarrows}[2][right]{\delimit[#1]{\Downarrow}{#2}}
%    \end{macrocode}
% \end{macro}
%
% \begin{macro}{\uparrows}
%    \begin{macrocode}
\newcommand{\uparrows}[2][right]{\delimit[#1]{\uparrow}{#2}}
%    \end{macrocode}
% \end{macro}
%
% \begin{macro}{\Uparrows}
%    \begin{macrocode}
\newcommand{\Uparrows}[2][right]{\delimit[#1]{\Uparrow}{#2}}
%    \end{macrocode}
% \end{macro}
%
% \begin{macro}{\updownarrows}
%    \begin{macrocode}
\newcommand{\updownarrows}[2][right]{\delimit[#1]{\updownarrow}{#2}}
%    \end{macrocode}
% \end{macro}
%
% \begin{macro}{\Updownarrows}
%    \begin{macrocode}
\newcommand{\Updownarrows}[2][right]{\delimit[#1]{\Updownarrow}{#2}}
%    \end{macrocode}
% \end{macro}
%
% \subsubsection{Shortcuts for Large Delimiters}
%
% Since the delimiters here are all ``large'' delimiters (requiring
% the use of \verb!\left! and \verb!\right!), these macros take no
% optional argument.
%
% \begin{macro}{\moustaches}
%    \begin{macrocode}
\newcommand{\moustaches}[2]{\lrpair[right]{moustache}{#2}}
%    \end{macrocode}
% \end{macro}
%
% \begin{macro}{\groups}
%    \begin{macrocode}
\newcommand{\groups}[2]{\lrpair[right]{group}{#2}}
%    \end{macrocode}
% \end{macro}
%
% \begin{macro}{\arrowverts}
%    \begin{macrocode}
\newcommand{\arrowverts}[2]{\delimit[right]{\arrowvert}{#2}}
%    \end{macrocode}
% \end{macro}
%
% \begin{macro}{\Arrowverts}
%    \begin{macrocode}
\newcommand{\Arrowverts}[2]{\delimit[right]{\Arrowvert}{#2}}
%    \end{macrocode}
% \end{macro}
%
% \begin{macro}{\braceverts}
%    \begin{macrocode}
\newcommand{\braceverts}[2]{\delimit[right]{\bracevert}{#2}}
%    \end{macrocode}
% \end{macro}
%
%
% \section{Misc}
%
%
% \begin{macro}{\evalat}
%    \begin{macrocode}
\newcommand{\evalat}[3][right]{
  \@ifnextchar[
    {\lr@evalat{#1}{#2}{#3}}
    {\lr@evalat{#1}{#2}{#3}[]}%]
}
%    \end{macrocode}
% \end{macro}
%
%
% \begin{macro}{lr@evalat}
%    \begin{macrocode}
\def\lr@evalat#1#2#3[#4]{{
  \lr@lrsize{#1}
  \lr@left. #2 \lr@right|_{#3}^{#4}
}}
%    \end{macrocode}
% \end{macro}
%
%
% \begin{macro}{\setst}
%   Users can redefine this command to change the set builder notation
%   used in \verb!\sbuilder!.  Specifically, this macro defines the
%   "such that" symbol used in the notation.
%
%    \begin{macrocode}
\newcommand{\setst}{\;\lr@middle\vert\;}
%    \end{macrocode}
% \end{macro}
%
%
% \begin{macro}{\sbuilder}
%    \begin{macrocode}
\newcommand{\sbuilder}[3][right]
{{
  \lr@lrmsize{#1}
  \lr@left\{ #2 \setst #3 \lr@right\}
}}
%    \end{macrocode}
% \end{macro}

%
%
%    \begin{macrocode}
\EndPackage
%</package>
%    \end{macrocode}
%
%
% \CheckSum{204}
%
% \CharacterTable
%  {Upper-case    \A\B\C\D\E\F\G\H\I\J\K\L\M\N\O\P\Q\R\S\T\U\V\W\X\Y\Z
%   Lower-case    \a\b\c\d\e\f\g\h\i\j\k\l\m\n\o\p\q\r\s\t\u\v\w\x\y\z
%   Digits        \0\1\2\3\4\5\6\7\8\9
%   Exclamation   \!     Double quote  \"     Hash (number) \#
%   Dollar        \$     Percent       \%     Ampersand     \&
%   Acute accent  \'     Left paren    \(     Right paren   \)
%   Asterisk      \*     Plus          \+     Comma         \,
%   Minus         \-     Point         \.     Solidus       \/
%   Colon         \:     Semicolon     \;     Less than     \<
%   Equals        \=     Greater than  \>     Question mark \?
%   Commercial at \@     Left bracket  \[     Backslash     \\
%   Right bracket \]     Circumflex    \^     Underscore    \_
%   Grave accent  \`     Left brace    \{     Vertical bar  \|
%   Right brace   \}     Tilde         \~}
%
% \endinput
% Local Variables:
% mode: doctex
% TeX-master: t
% End:
