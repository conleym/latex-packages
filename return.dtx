% \iffalse meta-comment
%<*internal>
\iffalse
%</internal>
%<*readme>
----------------------------------------------------------------
return --- Return a value to the enclosing group
Author:  Mike Conley
E-mail:  conasdf@gmail.com
License: Released under the LaTeX Project Public License v1.3c or later
See:     http://www.latex-project.org/lppl.txt
----------------------------------------------------------------
%</readme>
%<*internal>
\fi
\def\nameofplainTeX{plain}
\ifx\fmtname\nameofplainTeX\else
  \expandafter\begingroup
\fi
%</internal>
%<*install>
\input docstrip.tex
\keepsilent
\askforoverwritefalse
\preamble
----------------------------------------------------------------
return --- Return a value to the enclosing group
Author:  Mike Conley
E-mail:  conasdf@gmail.com
License: Released under the LaTeX Project Public License v1.3c or later
See:     http://www.latex-project.org/lppl.txt
----------------------------------------------------------------

\endpreamble
\postamble

Copyright (C) 2005 by Mike Conley <conasdf@gmail.com>

This work may be distributed and/or modified under the
conditions of the LaTeX Project Public License (LPPL), either
version 1.3c of this license or (at your option) any later
version.  The latest version of this license is in the file:

http://www.latex-project.org/lppl.txt

This work is "maintained" (as per LPPL maintenance status) by
Mike Conley.

This work consists of the file return.dtx and a Makefile.
Running "make" generates the derived files README, return.pdf and return.sty.
Running "make inst" installs the files in the user's TeX tree.
Running "make install" installs the files in the local TeX tree.

\endpostamble

\usedir{tex/latex/return}
\generate{
  \file{\jobname.sty}{\from{\jobname.dtx}{package}}
}
%</install>
%<install>\endbatchfile
%<*internal>
\usedir{source/latex/return}
\generate{
  \file{\jobname.ins}{\from{\jobname.dtx}{install}}
}
\nopreamble\nopostamble
\usedir{doc/latex/return}
\generate{
  \file{README-\jobname.txt}{\from{\jobname.dtx}{readme}}
}
\ifx\fmtname\nameofplainTeX
  \expandafter\endbatchfile
\else
  \expandafter\endgroup
\fi
%</internal>
% \fi
%
% \iffalse
%<*driver>
\ProvidesFile{return.dtx}
%</driver>
%<package>\NeedsTeXFormat{LaTeX2e}[1999/12/01]
%<package>\RequirePackage{package}
%<package>\Package{return}
%<*package>
    [2005/08/30 v1.00 Return a value to the enclosing group]
%</package>
%<*driver>
\documentclass{ltxdoc}
\usepackage[a4paper,margin=25mm,left=50mm,nohead]{geometry}
\usepackage[numbered]{hypdoc}
\usepackage{\jobname}
\EnableCrossrefs
\CodelineIndex
\RecordChanges
\begin{document}
  \DocInput{\jobname.dtx}
\end{document}
%</driver>
% \fi
%
% \GetFileInfo{\jobname.dtx}
%
%
%\title{\textsf{return} --- Return a value to the enclosing group\thanks{This file
%   describes version \fileversion, last revised \filedate.}
%}
%\author{Mike Conley\thanks{E-mail: conasdf@gmail.com}}
%\date{Released \filedate}
%
%\maketitle
%
%\changes{v1.00}{2005/08/30}{First public release}
%
%
% \section{Usage}
%
%
% This is a simple package defining two macros: |\return| and |\dreturn|.
% |\return|\marg{macro body} uses |\xdef| to define a global macro that will be
% expanded immediately after the current group is ended.  Similarly, |\dreturn|
% uses |\gdef|. These are useful for propogating values up one level, rather
% than globally.
%
% The difference between the two is clear in view of the difference between
% |\xdef| (defined as |\global\edef|) and |\gdef| (defined as |\global\def|),
% i.e., the difference between |\edef| and |\def|: |\edef| expands at definition
% time, whereas |\def| delays expansion until use.
%
%
% \section{Implementation}
%
%    \begin{macrocode}
%<*package>

%    \end{macrocode}
% \begin{macro}{\return}
% This simply defines a ``private'' macro, |\return@return| (using |\xdef|),
% to be expanded the current group is closed.
%    \begin{macrocode}
\newcommand{\return}{
  \aftergroup\return@return
  \xdef\return@return
}
%    \end{macrocode}
% \end{macro}
%
%
% \begin{macro}{\greturn}
% The same as |\return|, but using |\gdef| instead of |\xdef|.
%    \begin{macrocode}
\newcommand{\greturn}{
  \aftergroup\return@return
  \gdef\return@return
}
%    \end{macrocode}
% \end{macro}
%
%
%    \begin{macrocode}
\EndPackage
%</package>
%    \end{macrocode}
%
%
% \CheckSum{0}
%
% \CharacterTable
%  {Upper-case    \A\B\C\D\E\F\G\H\I\J\K\L\M\N\O\P\Q\R\S\T\U\V\W\X\Y\Z
%   Lower-case    \a\b\c\d\e\f\g\h\i\j\k\l\m\n\o\p\q\r\s\t\u\v\w\x\y\z
%   Digits        \0\1\2\3\4\5\6\7\8\9
%   Exclamation   \!     Double quote  \"     Hash (number) \#
%   Dollar        \$     Percent       \%     Ampersand     \&
%   Acute accent  \'     Left paren    \(     Right paren   \)
%   Asterisk      \*     Plus          \+     Comma         \,
%   Minus         \-     Point         \.     Solidus       \/
%   Colon         \:     Semicolon     \;     Less than     \<
%   Equals        \=     Greater than  \>     Question mark \?
%   Commercial at \@     Left bracket  \[     Backslash     \\
%   Right bracket \]     Circumflex    \^     Underscore    \_
%   Grave accent  \`     Left brace    \{     Vertical bar  \|
%   Right brace   \}     Tilde         \~}
%
% \endinput
% Local Variables:
% mode: doctex
% TeX-master: t
% End:
